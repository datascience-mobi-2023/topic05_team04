% Options for packages loaded elsewhere
\PassOptionsToPackage{unicode}{hyperref}
\PassOptionsToPackage{hyphens}{url}
%
\documentclass[
  11pt,
]{article}
\usepackage{amsmath,amssymb}
\usepackage{setspace}
\usepackage{iftex}
\ifPDFTeX
  \usepackage[T1]{fontenc}
  \usepackage[utf8]{inputenc}
  \usepackage{textcomp} % provide euro and other symbols
\else % if luatex or xetex
  \usepackage{unicode-math} % this also loads fontspec
  \defaultfontfeatures{Scale=MatchLowercase}
  \defaultfontfeatures[\rmfamily]{Ligatures=TeX,Scale=1}
\fi
\usepackage{lmodern}
\ifPDFTeX\else
  % xetex/luatex font selection
\fi
% Use upquote if available, for straight quotes in verbatim environments
\IfFileExists{upquote.sty}{\usepackage{upquote}}{}
\IfFileExists{microtype.sty}{% use microtype if available
  \usepackage[]{microtype}
  \UseMicrotypeSet[protrusion]{basicmath} % disable protrusion for tt fonts
}{}
\makeatletter
\@ifundefined{KOMAClassName}{% if non-KOMA class
  \IfFileExists{parskip.sty}{%
    \usepackage{parskip}
  }{% else
    \setlength{\parindent}{0pt}
    \setlength{\parskip}{6pt plus 2pt minus 1pt}}
}{% if KOMA class
  \KOMAoptions{parskip=half}}
\makeatother
\usepackage{xcolor}
\usepackage[margin=2.3cm]{geometry}
\usepackage{graphicx}
\makeatletter
\def\maxwidth{\ifdim\Gin@nat@width>\linewidth\linewidth\else\Gin@nat@width\fi}
\def\maxheight{\ifdim\Gin@nat@height>\textheight\textheight\else\Gin@nat@height\fi}
\makeatother
% Scale images if necessary, so that they will not overflow the page
% margins by default, and it is still possible to overwrite the defaults
% using explicit options in \includegraphics[width, height, ...]{}
\setkeys{Gin}{width=\maxwidth,height=\maxheight,keepaspectratio}
% Set default figure placement to htbp
\makeatletter
\def\fps@figure{htbp}
\makeatother
\setlength{\emergencystretch}{3em} % prevent overfull lines
\providecommand{\tightlist}{%
  \setlength{\itemsep}{0pt}\setlength{\parskip}{0pt}}
\setcounter{secnumdepth}{5}
\ifLuaTeX
  \usepackage{selnolig}  % disable illegal ligatures
\fi
\IfFileExists{bookmark.sty}{\usepackage{bookmark}}{\usepackage{hyperref}}
\IfFileExists{xurl.sty}{\usepackage{xurl}}{} % add URL line breaks if available
\urlstyle{same}
\hypersetup{
  hidelinks,
  pdfcreator={LaTeX via pandoc}}

\author{}
\date{\vspace{-2.5em}}

\begin{document}

\setstretch{1.15}
\pagenumbering{gobble}
\begin{titlepage}
\centering
    {\Large Ruprecht-Karls University of Heidelberg\\
        Faculty of Engineering Sciences\\
        BSc Molecular Biotechnology\\}

    {\vspace{\stretch{3}}}


        {\Huge Drug viability screens for oncological and non-oncological treatments for breast cancer}

        

    

    \vspace{\stretch{3}}
    {\Large Data Science Project Summer Semester 2023}
    
    \vspace{\stretch{0.25}}
    
    {\Large Topic 5 Team 4}

    \vspace{\stretch{0.25}}
    
{\Large Submission date: 17.07.2023}

    \vspace{\stretch{0.75}}

    {\Large
        Luis Herfurth, Aaron Eidenmüller, Sharujan Suthakaran, Simon Westermann  
}

    \vspace{\stretch{1}}

\end{titlepage}

\newpage
\section*{Abstract}
Hier muss das Abstract eingefügt werden


\newpage
\tableofcontents

\newpage
\section*{Abbreviations}





\newpage
\pagenumbering{arabic}

\hypertarget{introduction}{%
\section{Introduction}\label{introduction}}

As breast cancer is the most common type of cancer in women, accounting
for approximately 2.3 million cases each year, treating breast cancer is
still a major objective in global research (Łukasiewicz \emph{et al.},
2021). ``Even though a significant number of cancers do not always need
to result in death, they significantly lower the quality of life and
require larger costs in general.'' (Łukasiewicz \emph{et al.}, 2021)
While therapies such as chemotherapy have significantly improved
outcomes, they are often accompanied by notable side effects. Therefore,
there is a critical need to discover new compounds that can enhance
treatment outcomes for breast cancer while minimizing adverse effects.
(Kumbhar \emph{et al.}, 2023) Nevertheless, the process of drug
development is lengthy and expensive, typically taking around 12-15
years and costing billions of dollars to bring a new compound to market
(Wouters \emph{et al.}, 2020). This challenge is further amplified in
the field of cancer, as cancer cells exhibit a heightened resistance to
treatment with estimations that only about 5\% of potential drugs enter
the clinical testing (Sleire \emph{et al.}, 2017). Consequently,
researching and developing new cancer treatments pose substantial risks
for pharmaceutical companies. Small companies and start-ups, in
particular, require significant investments to undertake such
endeavours. It is estimated that for every dollar a pharmaceutical
company invests in drug development, less than a dollar is returned,
resulting in minimal profitability and discouraging potential investors.
This issue is particularly detrimental to the development of medications
for rare diseases and those prevalent in developing countries, as the
financial incentives are even more limited (Pushpakom \emph{et al.},
2019). A potential solution to this problem is drug repurposing, which
involves utilizing already approved drugs for the treatment of other
diseases. By repurposing existing drugs, the cost can be significantly
reduced to around 300 million dollars, and the development time can be
greatly shortened, as these drugs have already undergone the necessary
regulatory processes. (Nosengo, 2016) By addressing these challenges
through drug repurposing, we can expedite the availability of new
treatment options for breast cancer and other diseases, making
substantial progress in improving patient outcomes while maximizing
resource efficiency. Drug repurposing can be approached through various
methods, including computational and experimental approaches. In this
research project, the chosen approach was a computational one,
specifically focusing on genetic associations. The objective was to
identify genes associated with the disease that could potentially serve
as targets for future drug development (Pushpakom \emph{et al.}, 2019).
To achieve this, treatment response data for cancer cells obtained using
the PRISM method, along with genetic data such as gene expression
profiles of the cancer cell lines, were collected. These data were
utilized to explore new opportunities for drug repurposing. In the
beginning, the treatments in the data sets are categorized on their
potential of inhibition. This data will then be used for analysis of
similarity between treatments as well as being the basis for a genetic
analysis of the data. Specifically, this research project emphasizes the
use of gene expression as an indicator for identifying genes that could
serve as new drug targets. With advancements in analysis techniques and
the increasing affordability of gene expression analysis, it has become
feasible to analyse gene expression for individual patients and identify
potential drugs for targeted treatment (Chawla \emph{et al.}, 2022). The
potential target genes are subsequently evaluated based on their gene
knockout scores, which indicate the extent to which the knockout of the
target gene inhibits cell growth significantly. Furthermore, predicting
the effectiveness of the possibly repurposed drugs depending on the
treatment dose is another objective of our research. As during the
preclinical and clinical testing phase drugs are evaluated on their
effective dose and their maximum dose, it is important to be able to
predict if a drug can have a beneficial effect in the permitted dose.\\
By leveraging the genetic association approach and focusing on gene
expression analysis as well as analysing the inhibiting treatments, this
research aims to identify promising candidates for drug repurposing and
pave the way for the development of new treatment options tailored to
specific patients.

\hypertarget{materials-and-methods}{%
\section{Materials and Methods}\label{materials-and-methods}}

\hypertarget{data}{%
\subsection{Data}\label{data}}

\hypertarget{prism-datasets}{%
\subsubsection{Prism Datasets}\label{prism-datasets}}

\textbf{Prism:} effect of the treatment (columns) on cell growth of the
cell lines (rows); includes drug, dosage and assay used

\textbf{Prism.treat:} for each treatment (rows) further information on
the treatment and drug

\textbf{Prism.cl:} contains information about the different celllines

\hypertarget{cellline-datasets}{%
\subsubsection{Cellline Datasets}\label{cellline-datasets}}

\textbf{Prism.exp:} contains levels of gene expression. Celllines (rows)
and genes (columns)

The copy number data frame contains the copy number of each gene for
each cell line used in the PRISM screen. Each row of the data frame
represents a specific cell line, while the columns correspond to
different genes. The row names and column names of the data frame are
assigned using the respective cell line IDs and gene designations. The
copy number of a gene refers to the number of copies present in the
genome of an individual cell. Theoretically, each gene has a copy number
of two, because one gene copy of each parent exists in the genome.
However, mutations can lead to variations in the copy numbers, resulting
in either higher or lower values than the expected two copies.

\textbf{Prism.snv:} marks mutation in the different celllines als
functional or nonfunctional to the cancer.

\textbf{Prism.achilles:} has information on how important a gene is for
cell survival. Was generated using knockdown celllines. Gene names
(rows) and celllines (colums) Lastly, a data frame containing the gene
knockout scores for the cancer cell lines. This means that each gene has
been silenced via CRISPR or another method and the resulting effect on
cell proliferation was tested. A low gene knockout score indicates shows
that the cell proliferation has been severly inhibited. Therefore the
importance of genes can be estimated by how low the gene knockout score
is.

\hypertarget{data-clean-upfiltering}{%
\subsection{Data clean up/Filtering}\label{data-clean-upfiltering}}

Show distributions after cleanup Abbildung für prism vorher nachher. Für
andere maybe im Clean up

Before performing the cleanup process, a subset of the data frame was
created specifically for the copy number data of the 22 breast cancer
cell lines. This subset was obtained by extracting the ID codes of the
breast cell lines from the comprehensive data frame that included
information on all cell lines used in the PRISM screen. The extracted ID
codes were saved in a vector, which was then utilized to construct a new
data frame containing the copy number information solely for the 22
breast cell lines. Next, both copy number data frames were checked for
missing values, and it was determined that no missing values were
present. Consequently, the data frames were deemed suitable for further
analysis and processing.

\hypertarget{dimension-reduction}{%
\subsection{Dimension reduction}\label{dimension-reduction}}

Uniform Manifold Approximation and Projection (UMAP) was utilized as a
method for reducing the dimensionality of data in this study. UMAP is a
non-linear technique designed to retain local structure and capture
intricate relationships within high-dimensional datasets. The `umap'
function from the `umap' package in R was employed to apply UMAP. The
method involves several steps: initially, pairwise distances are
computed between data points using a selected distance metric.
Subsequently, a neighborhood graph is constructed by connecting each
point to its closest neighbors. UMAP then estimates a fuzzy-topological
representation of the data based on this graph, capturing the
connectivity strength via a fuzzy simplicial set. To find a
low-dimensional representation that minimizes the inconsistency between
pairwise similarities in the original high-dimensional space and the
reduced space, an optimization process driven by stochastic gradient
descent is conducted. This iterative optimization accounts for
attractive and repulsive forces, ultimately yielding a lower-dimensional
embedding that represents the data. The `n\_neighbors' parameter
specifies the number of nearest neighbors considered during graph
construction, and the `min\_dist' parameter governs the minimum distance
between points in the low-dimensional embedding. The resulting UMAP
embedding maintains local structure and provides a visually
interpretable representation, facilitating data exploration and
analysis.

\hypertarget{results}{%
\section{Results}\label{results}}

First include positiv results; if space is left include negativ results:
UMAP, K means clustering, promoting drugs describe goal, describe
process, describe outcome

\hypertarget{gene-search-engine}{%
\subsection{Gene search engine}\label{gene-search-engine}}

Analysing the main datasets led to many individual data formats such as
data frames, lists etc.; which contain relevant information gathered by
our code. The aim of the function of the search engine was to quickly
search for a gene of interest and display its attached values out of our
main cell line datasets (prism.exp, prism.achilles, prism.cnv) for a
more simple process of gene analysis. Based on this code, applications
like looking for suitable treatments in prism.treat, loops which undergo
printing every gene and their attached values in breast cancer celllines
and the first approach for a relevant final data frame were realized.
The conclusive outcome of this engine development was a final modified
search engine, in which one can type in the gene of interest and it
prints every relevant value or information to get a decisive overview
for drug repurposing applications.

\hypertarget{list-of-inhibitory-drugs}{%
\subsection{List of inhibitory drugs}\label{list-of-inhibitory-drugs}}

Results von Data clean up und filtering. Goal: List of Inhibitory Drugs;
Outcome: List of Inhibitory Drugs Bilder vergleich liste vergleich ohne
threshold und mit threshold Maybe oncological drugs rein screenen

\hypertarget{umap}{%
\subsection{UMAP}\label{umap}}

\hypertarget{gene-analysis}{%
\subsection{Gene analysis}\label{gene-analysis}}

\hypertarget{correlation-analysis}{%
\subsubsection{Correlation analysis}\label{correlation-analysis}}

The initial step in gene analysis was examining the correlation between
different variables to determine if there were any connections between
them. This approach aimed to identify genes that could potentially serve
as indicators of a drug's effectiveness. To begin, a correlation
analysis between the gene expression data in prism.exp and the gene copy
number data in prism.cnv was conducted. The method used was the Pearson
correlation. Our hypothesis was that the copy number would correlate
highly with gene expression. This was important to see if the number of
gene copies can be an indicator for gene expression. Figure XX presents
the histogram of the correlation calculations, which demonstrates that,
for the majority of genes, a positive correlation indeed exists. The
mean correlation was approximately 0.325, with a median of 0.345. Yet,
the correlation is not as high as was expected. Around 15.23\% of the
correlations are negative. In the subsequent step, we constructed a
correlation matrix using Pearson correlation to examine the relationship
between gene expression data from prism.exp and treatment response data
from prism.treat. This correlation matrix allowed us to further refine
our understanding of which genes were associated with specific
treatments. The resulting matrix contains correlations between 18,805
genes and 1,395 treatments. The data was then used in the further gene
analysis.

\hypertarget{statistical-testing-of-important-genes}{%
\subsubsection{Statistical testing of important
genes}\label{statistical-testing-of-important-genes}}

A threshold of absolute correlation greater than 0.75 was applied to
select genes for subsequent analysis. This threshold was selected as it
is the mean of the highest absolute correlation for each treatment. The
filtering resulted in selection of 3925 genes out of the total 18119
genes in the prism.achilles dataset. These genes were then sorted based
on their prism.achilles scores being lower than 0 and then a one-sided
Wilcoxon rank sum test with significance level 0.05 was performed to
assess if their mean scores were significantly lower than those of other
lineages. As previous Shapiro-Wilk-Test showed the data is not normally
distributed and therefore a non-parametric test was needed. The p-values
were adjusted using the False Discovery Rate (FDR) correction. One gene
showed a significant difference, SDHC with a p-value of 0.025.

\hypertarget{dataframe-for-targets-involving-genes}{%
\subsubsection{Dataframe for targets involving
genes}\label{dataframe-for-targets-involving-genes}}

Moreover, the 3925 preselected genes were compiled into a data frame for
further research. Genes with a mean achilles score below -1 for breast
cell lines were included in the final data frame. 108 preselected genes
fulfilled this criterion. This data frame includes information such as
the frequency of high absolute correlation with a treatment, mean
expression score, mean prism score and the p-value from the Wilcoxon
rank sum test. Furthermore, it provides additional details on gene
mutations in breast cancer cell lines, treatments targeting the gene,
and its association with any cancer hallmarks. This data frame concludes
our gene analysis as it combines our gene selection process with general
information we gather about the genes along the way. Two genes showed a
high frequency amongst treatments, mTOR and AURKA. These genes were
associated with more than 20 treatments during the analysis.

\hypertarget{linear-regression}{%
\subsection{Linear regression}\label{linear-regression}}

Perform drug by drug to avoid weird plot; For every drug one linear
regression, R² Value and with those showing, that many of them are very
good. Prediction model for concentration and drug name. Plot um das
gnaze zu veranschaulichen; Goal: Regression/Prediction model; Outcome:
Regression/Prediction model

\hypertarget{discussion}{%
\section{Discussion}\label{discussion}}

gene search engine: An informative gene search engine is a promising
advancement in the field of genomic data analysis. Regarding large
screenings, utilizing a specific profile as an input to the engine,
offers accurate results, enabling other researchers to identify genes
and their possible association to another treatment. Respectively, one
could have got a possible important gene through literature analysis or
comparing every meaningful value in our big database manually, but it is
time-consuming and one could lose the overview. The demonstrated
scalability and function of our engine does not display the current
standard of such engines with a wide range of abilities, being GEMINI
and Sigcom LINCS. While further improvements may be considered, our gene
search engine occurred to be a usable tool for exploring gene-associated
data for an overview and contributes to helping researchers to find
their specific gene-treatment interaction.

The Pearson correlation analysis of the gene expression and the copy
number validated our hypothesis partially. A trend of high positive
correlation can be seen on the Fig. XX as shown in the results. This
indicates that if there are more copies of a gene it is likely that the
gene expression is upregulated for this gene as well. But as the graph
shows, this is not always the case as 15,23\% of the correlations are
negative. This might be the case since gene regulation isn't as simple
as only the gene copy. Other regulatory units that are on another part
of the genome could be involved. It could also be that after some point
the higher the copy number gets, the less effect it has on gene
expression. Further research across cancer cell lines shows similar
results (Shao \emph{et al.}, 2019). They conclude that there is a mostly
positive correlation between copy number and gene expression as well.

Correlation analysis of breast cancer concluded one gene that is
particularly interesting, SDHC. The correlation score of SDHC is -0.751
which means that a high gene expression indicates low treatment score
and therefore a good treatment response. The Wilcoxon-rank-sum test
showed that the SDHC gene has a significantly lower mean knockout score
compared to the other lineages. This suggest that SDHC has particular
importance in the breast cancer cell lines in our research project. The
gene SDHC codes for the succinate dehydrogenase complex subunit c, a
subunit of the succinate dehydrogenase important for anchoring and
stabilization. The succinate dehydrogenase is part of the respiratory
chain in mitochondria. Potential risks of mutations are a disbalance in
the respiratory chain and oxidative stress (Cerqua \emph{et al.}, 2021).
SDHC has been linked in the correlation analysis to only one treatment,
known as tegafur. This drug targets the thymidylate synthase (Lee
\emph{et al.}, 2015). Using the prediction model from this research
project shows that tegafur response in breast cancer can be predicted
quite accurately. Literature research have shown that tegafur already
has been used on some cases in breast cancer with positive results. Yet,
it is unclear how effective the treatment is and if it opens a new
potential for targeted application if a high gene expression in patients
can be detected. As it has already been tested if tegafur has a
potential effect in breast cancer, the objective of finding novel
applications for treatments has not been met.

Our final dataframe revealed results regarding the importance of mTOR in
breast cancer cell lines. A mean importance value of -1.268 emphasizes
that the knock-out of mTOR could play a crucial role in breast cancer
development. This matches with existing literature, which has shown that
dysregulation of mTOR in the PAM pathway is associated with increased
tumor growth in breast cancer (Zhu \emph{et al.}, 2022). The mechanistic
target of rapamycin (mTOR) encodes a protein kinase, functioning as a
central regulator of essential cellular pathways involved in growth,
proliferation and survival (Zhu \emph{et al.}, 2022). The regulatory
function of mTOR makes it a possible cause for various cancer types when
mutated and consequently an attractive target for therapeutic
interventions. Moreover, we obtained a mean expression level value of
4.736, which indicates a high expression as the average of all genes are
2.729. This supports the previous literature findings, which suggest
that mTOR is commonly overexpressed in breast cancer, further stressing
its potential oncogenic involvement (Zhu \emph{et al.}, 2022). In
addition to that, we examined the copy number variation of the mTOR gene
in breast cancer cell lines and the mean value of 0.912 could display
genomic instability due to partial deletion of the gene as the copy
number is lower than two. This discovery would further promote the role
of mTOR in cancer development. Including the hypothesis, we made out the
three values, possible treatments should be able to downregulate the
function of the mTOR gene. In fact, the associated treatments presented
in the final dataframe are mTOR inhibitors. Finally, our results
reinforce the significance of mTOR in breast cancer. However, targeting
mTOR in breast cancer should be investigated to develop safe treatment
strategies. Nevertheless, our study about mTOR is similar to the
groundwork findings, consequently suited for future research.

The basis of our gene analysis is the correlation analysis of treatment
response and gene expression. The idea was that high absolute
correlation could show genes that are associated with breast cancer.
This means that a particular gene could be linked to a treatment.
However, this does not imply causality meaning that the treatment could
possibly not be affected directly by the treatment. The next selection
process was on the importance of the genes via gene knockout score. The
findings conclude genes and treatments that have been associated with
cancer or further breast cancer but some only slightly. Novel
opportunities for drug repurposing have not been found. In hindsight,
this approach may not be optimal to further decrease the number of
potential targets. As stated before, correlation of gene expression does
not imply the direct causality between the treatment and the gene
expression. So, looking at the importance of a gene via gene knockout
score does not show that a gene could be a good target.

\hypertarget{references}{%
\section{References}\label{references}}

\hypertarget{appendix}{%
\section{Appendix}\label{appendix}}

\end{document}
