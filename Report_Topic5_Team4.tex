% Options for packages loaded elsewhere
\PassOptionsToPackage{unicode}{hyperref}
\PassOptionsToPackage{hyphens}{url}
%
\documentclass[
  11pt,
]{article}
\usepackage{amsmath,amssymb}
\usepackage{setspace}
\usepackage{iftex}
\ifPDFTeX
  \usepackage[T1]{fontenc}
  \usepackage[utf8]{inputenc}
  \usepackage{textcomp} % provide euro and other symbols
\else % if luatex or xetex
  \usepackage{unicode-math} % this also loads fontspec
  \defaultfontfeatures{Scale=MatchLowercase}
  \defaultfontfeatures[\rmfamily]{Ligatures=TeX,Scale=1}
\fi
\usepackage{lmodern}
\ifPDFTeX\else
  % xetex/luatex font selection
\fi
% Use upquote if available, for straight quotes in verbatim environments
\IfFileExists{upquote.sty}{\usepackage{upquote}}{}
\IfFileExists{microtype.sty}{% use microtype if available
  \usepackage[]{microtype}
  \UseMicrotypeSet[protrusion]{basicmath} % disable protrusion for tt fonts
}{}
\makeatletter
\@ifundefined{KOMAClassName}{% if non-KOMA class
  \IfFileExists{parskip.sty}{%
    \usepackage{parskip}
  }{% else
    \setlength{\parindent}{0pt}
    \setlength{\parskip}{6pt plus 2pt minus 1pt}}
}{% if KOMA class
  \KOMAoptions{parskip=half}}
\makeatother
\usepackage{xcolor}
\usepackage[margin=2.3cm]{geometry}
\usepackage{graphicx}
\makeatletter
\def\maxwidth{\ifdim\Gin@nat@width>\linewidth\linewidth\else\Gin@nat@width\fi}
\def\maxheight{\ifdim\Gin@nat@height>\textheight\textheight\else\Gin@nat@height\fi}
\makeatother
% Scale images if necessary, so that they will not overflow the page
% margins by default, and it is still possible to overwrite the defaults
% using explicit options in \includegraphics[width, height, ...]{}
\setkeys{Gin}{width=\maxwidth,height=\maxheight,keepaspectratio}
% Set default figure placement to htbp
\makeatletter
\def\fps@figure{htbp}
\makeatother
\setlength{\emergencystretch}{3em} % prevent overfull lines
\providecommand{\tightlist}{%
  \setlength{\itemsep}{0pt}\setlength{\parskip}{0pt}}
\setcounter{secnumdepth}{5}
\ifLuaTeX
  \usepackage{selnolig}  % disable illegal ligatures
\fi
\IfFileExists{bookmark.sty}{\usepackage{bookmark}}{\usepackage{hyperref}}
\IfFileExists{xurl.sty}{\usepackage{xurl}}{} % add URL line breaks if available
\urlstyle{same}
\hypersetup{
  hidelinks,
  pdfcreator={LaTeX via pandoc}}

\author{}
\date{\vspace{-2.5em}}

\begin{document}

\setstretch{1.15}
\pagenumbering{gobble}
\begin{titlepage}
\centering
    {\Large Ruprecht-Karls University of Heidelberg\\
        Faculty of Engineering Sciences\\
        BSc Molecular Biotechnology\\}

    {\vspace{\stretch{3}}}


        {\Huge Drug viability screens for oncological and non-oncological treatments for breast cancer}

        

    

    \vspace{\stretch{3}}
    {\Large Data Science Project Summer Semester 2023}
    
    \vspace{\stretch{0.25}}
    
    {\Large Topic 5 Team 4}

    \vspace{\stretch{0.25}}
    
{\Large Submission date: 17.07.2023}

    \vspace{\stretch{0.75}}

    {\Large
        Luis Herfurth, Aaron Eidenmüller, Sharujan Suthakaran, Simon Westermann  
}

    \vspace{\stretch{1}}

\end{titlepage}

\newpage
\section*{Abstract}
Hier muss das Abstract eingefügt werden


\newpage
\tableofcontents

\newpage
\section*{Abbreviations}





\newpage
\pagenumbering{arabic}

\hypertarget{introduction}{%
\section{Introduction}\label{introduction}}

\hypertarget{materials-and-methods}{%
\section{Materials and Methods}\label{materials-and-methods}}

\hypertarget{data}{%
\subsection{Data}\label{data}}

\hypertarget{prism-datasets}{%
\subsubsection{Prism Datasets}\label{prism-datasets}}

\textbf{Prism:} effect of the treatment (columns) on cell growth of the
cell lines (rows); includes drug, dosage and assay used

\textbf{Prism.treat:} for each treatment (rows) further information on
the treatment and drug

\textbf{Prism.cl:} contains information about the different celllines\\
\emph{if we search after ``breast'' in the column lineage we get our 22
celllines}

\hypertarget{cellline-datasets}{%
\subsubsection{Cellline Datasets}\label{cellline-datasets}}

\textbf{Prism.exp:} contains levels of gene expression. Celllines (rows)
and genes (columns)

\textbf{Prism.cnv:} contains copy number levels of genes. Normal is CN =
2. Gene names (rows) and celllines (columns)

\textbf{Prism.snv:} marks mutation in the different celllines als
functional or nonfunctional to the cancer.

\textbf{Prism.achilles:} has information on how important a gene is for
cell survival. Was generated using knockdown celllines. Gene names
(rows) and celllines (colums)

\hypertarget{data-clean-upfiltering}{%
\subsection{Data clean up/Filtering}\label{data-clean-upfiltering}}

Show distributions after cleanup Abbildung für prism vorher nachher. Für
andere maybe im Clean up

\hypertarget{dimension-reduction}{%
\subsection{Dimension reduction}\label{dimension-reduction}}

UMAP -\textgreater{} PCA wird rausgelassen; Plot den Luis gemacht hat.
Ist nicht geclustert, aber man kann erkennen, dass die MOA sich in den
gleichen areas aufhalten

\hypertarget{results}{%
\section{Results}\label{results}}

First include positiv results; if space is left include negativ results:
UMAP, K means clustering, promoting drugs describe goal, describe
process, describe outcome

\hypertarget{gene-search-engine}{%
\subsection{Gene search engine}\label{gene-search-engine}}

Goal: Arbeitsvereinfachung; Outcome: Overview over data Für Präsentation
als Visualisierungstool pitchen Maybe Website,
\textbf{\emph{Discussion}}

Analysing the main datasets led to many individual data formats such as
data frames, lists etc.; which contain relevant information gathered by
our code. \textbf{\emph{In general, one could get a possible important
gene through literature analysis or comparing every meaningful value in
our database manually, which is time-consuming and one could lose
track.}} The aim of the function of the search engine was to quickly
search for a gene of interest and display its attached values out of our
main cell line datasets (prism.exp, prism.achilles, prism.cnv) for a
more simple process of gene analysis. Based on this code, applications
like looking for suitable treatments in prism.treat, loops which undergo
printing every gene and their attached values in breast cancer celllines
and the first approach for a relevant final data frame were realized.
The conclusive outcome of this engine development was a final modified
search engine, in which one can type in the gene of interest and it
prints every relevant value or information to get a decisive overview
for drug repurposing applications.

\hypertarget{list-of-inhibitory-drugs}{%
\subsection{List of inhibitory drugs}\label{list-of-inhibitory-drugs}}

Results von Data clean up und filtering. Goal: List of Inhibitory Drugs;
Outcome: List of Inhibitory Drugs Bilder vergleich liste vergleich ohne
threshold und mit threshold Maybe oncological drugs rein screenen

\#UMAP and

\hypertarget{gene-analysis}{%
\subsection{Gene analysis}\label{gene-analysis}}

\hypertarget{correlation-analysis}{%
\subsubsection{Correlation analysis}\label{correlation-analysis}}

treatment response / gene expression; Goal: finding relevant genes;
Outcome: giant data frame -\textgreater{} used for further work

copy number / gene expression Goal: looking if hypothesis correct;
Outcome: Histogram of correlations

The initial step in gene analysis was examining the correlation between
different variables to determine if there were any connections between
them. This approach aimed to identify genes that could potentially serve
as indicators of a drug's effectiveness. To begin, a correlation
analysis between the gene expression data in prism.exp and the gene copy
number data in prism.cnv was conducted. The method used was the Pearson
correlation. Our hypothesis was that the copy number would correlate
highly with gene expression, as higher copy numbers typically lead to
increased transcription. Figure XX presents the histogram of the
correlation calculations, which demonstrates that, for the majority of
genes, a positive correlation indeed exists. The mean correlation was
approximately 0.325, with a median of 0.345. Yet, the correlation is not
as high as was expected. In the subsequent step, we constructed a
correlation matrix using Pearson correlation to examine the relationship
between gene expression data from prism.exp and treatment response data
from prism.treat. This correlation matrix allowed us to further refine
our understanding of which genes were associated with specific
treatments. The resulting matrix contains correlations between 18,805
genes and 1,395 treatments. The data was then used in the further gene
analysis.

\hypertarget{statistical-testing-of-important-genes}{%
\subsubsection{Statistical testing of important
genes}\label{statistical-testing-of-important-genes}}

Test wich of the found genes are for breast cancer of interest Goal:
find out which one are negativ, which ones are lower than other
lineages; Outcome: 2 genes

A threshold of absolute correlation greater than 0.75 was applied to
select genes for subsequent analysis. This threshold was selected as it
is the mean of the highest absolute correlation for each treatment. The
filtering resulted in selection of 3925 genes out of the total 18119
genes in the prism.achilles dataset. These genes were then sorted based
on their prism.achilles scores being lower than 0 and then a one-sided
Wilcoxon rank sum test with significance level 0.05 was performed to
assess if their mean scores were significantly lower than those of other
lineages. As previous Shapiro-Wilk-Test showed the data is not normally
distributed and therefore a non-parametric test was needed. The p-values
were adjusted using the False Discovery Rate (FDR) correction. One gene
showed a significant difference, SDHC with a p-value of 0.025.

\hypertarget{dataframe-for-targets-involving-genes}{%
\subsubsection{Dataframe for targets involving
genes}\label{dataframe-for-targets-involving-genes}}

mean of data frames. Threshold for what genes are relevant. Used indings
from correlation tests Goal: finding interesting genes; Outcome: Data
frame with many genes -\textgreater{} 48 genes data set with filtering
after gene knockout score

Moreover, the 3925 preselected genes were compiled into a data frame for
further research. Genes with a mean achilles score below -1 for breast
cell lines were included in the final data frame. 108 preselected genes
fulfilled this criterion. This data frame includes information such as
the frequency of high absolute correlation with a treatment, mean
prism.exp score, mean prism.achilles score and the p-value from the
Wilcoxon rank sum test. Furthermore, it provides additional details on
gene mutations in breast cancer cell lines, treatments targeting the
gene, and its association with any cancer hallmarks. This data frame
concludes our gene analysis as it combines our gene selection process
with general information we gather about the genes along the way. Two
genes showed a high frequency amongst treatments, mTOR and AURKA. These
genes were associated with more than 20 treatments during the analysis.

\hypertarget{linear-regression}{%
\subsection{Linear regression}\label{linear-regression}}

Perform drug by drug to avoid weird plot; For every drug one linear
regression, R² Value and with those showing, that many of them are very
good. Prediction model for concentration and drug name. Plot um das
gnaze zu veranschaulichen; Goal: Regression/Prediction model; Outcome:
Regression/Prediction model

\hypertarget{discussion}{%
\section{Discussion}\label{discussion}}

Search for papers mentioning certain genes found in the targets of the
inhibitory drugs or

\hypertarget{references}{%
\section{References}\label{references}}

\hypertarget{appendix}{%
\section{Appendix}\label{appendix}}

\end{document}
